%! Author = huanglinxian
%! Date = 11/23/24

% Preamble
\documentclass[11pt]{article}

% Packages
\usepackage{amsmath}

\title{FINC 305 LN1 - Math Review}
\author{Linxian Huang}
\date{\today}

% Document
\begin{document}

\maketitle
    Linear algebra forms the backbone of most of  techniques used in financial modeling, providing the tools necessary to understand 
    and manipulate large datasets and complex systems. At its core, financial modeling often involves operations with matrices and vectors,
     making linear algebra indispensable. \\

\section{Vectors}

A \textit{vector} is an ordered finite list of numbers. Vectors are usually written as vertical arrays, surrounded by square or curved bracket, 
as in \\

\[
\begin{bmatrix}
-1.1 \\ 
0.0 \\ 
3.6 \\ 
-7.2
\end{bmatrix}
\quad \text{or} \quad
\begin{pmatrix}
-1.1 \\ 
0.0 \\ 
3.6 \\ 
-7.2
\end{pmatrix}.
\]

\textit{Element} is the value that included in the vector. The \textit{size} (or \textit{dimensions}, \textit{length}) is the numbers of elements 
that a vectors contains. For example, previous vector is 4-vector. 


\section{matrices}

A \textit{matrix} can be treated as a combination of multiple vectors with same size. For example, 

\[
\mathbf{A} =
\begin{bmatrix}
0 & 1 & -2.3 & 0.1 \\
1.3 & 4 & -0.1 & 0 \\
4.1 & -1 & 0 & 1.7
\end{bmatrix}.
\]
Based on the matrix example, matrix A is constituted by 4 (numbers of columns) 3-vectors (numbers of rows).  \\

In conclusion, a matrix is a 2-D array of numbers: A matrix with real-valued entries, \( m \) rows, and \( n \) columns.
\( A_{ij} \) denotes the value in the matrix in the \( i \)-th row and \( j \)-th column. Values included in the matrix should 
be real value,  which is 

\[
A \in \mathbb{R}^{m \times n}
\]


\section{Operation Reviews}

    \subsection{Matrix Addition}

    \[
(\mathbf{A} + \mathbf{B})_{ij} = a_{ij} + b_{ij}
\]

For example:
    \[
\mathbf{A} + \mathbf{B} =
\begin{bmatrix}
a_{11} & a_{12} \\
a_{21} & a_{22}
\end{bmatrix}
+
\begin{bmatrix}
b_{11} & b_{12} \\
b_{21} & b_{22}
\end{bmatrix}
=
\begin{bmatrix}
a_{11} + b_{11} & a_{12} + b_{12} \\
a_{21} + b_{21} & a_{22} + b_{22}
\end{bmatrix}
\]
\\
    Two matrices cannot be added unless they have the same dimensions.
For example, a 2 * 3 matrix (2 rows and 3 columns) and a 3 * 4 matrix (3 rows and 4 columns) cannot be added because their dimensions do not match.
    Matrix addition requires that the number of rows and columns in both matrices be identical so that corresponding elements can be added element-wise.


    \section{Practice}
    1. Let
\[
A =
\begin{pmatrix}
1 & -1 \\
2 & 1
\end{pmatrix}
\quad \text{and} \quad
B =
\begin{pmatrix}
-1 & 1 \\
0 & -3
\end{pmatrix}.
\]

Find \( A + B \), \( 3B \), \( -2B \), \( A + 2B \), \( A - B \), \( B - A \).

\end{document}